\documentclass[portuguese,12pt,a4paper]{book}
\usepackage[latin1]{inputenc}
\usepackage[T1]{fontenc}
\usepackage{graphicx}
\usepackage{mathtools}
\usepackage{amssymb}
\usepackage{amsthm}
\usepackage{babel}
\usepackage[colorlinks=true]{hyperref}
\usepackage{enumitem}
%\usepackage{biblatex}
%\addbibresource{bibliografia.bib}

\title{A teoria dos conjuntos e os fundamentos da matem�tica - Solu��es}

\newcommand{\cqd}{\hfill $\square$}
\newenvironment{solucao}[1][]{\noindent\textbf{Solu��o#1:} }{\cqd}
\newcounter{ex}
\newtheorem{exercicio}[ex]{Exerc�cio}

\begin{document}
	
	\maketitle
	\tableofcontents
	
	\include{capitulos/cap1}
	\chapter{A linguagem da Teoria dos Conjuntos}

\setcounter{ex}{0}

\begin{exercicio}
	Usando a linguagem de primeira ordem da teoria de conjuntos, escreva f�rmulas para representar as seguintes frases.
	\begin{enumerate}[label=(\alph{*})]
		\item N�o existe o conjunto de todos os conjuntos.
		\item Existe um �nico conjunto vazio.
		\item x � um conjunto unit�rio
		\item Existe um conjunto que tem como elemento apenas o conjunto vazio
		\item y � o conjunto dos subconjuntos de x 
	\end{enumerate}
\end{exercicio}

\begin{solucao}
	\begin{enumerate}[label=(\alph{*})]
		\item $\neg \exists x \forall y (y \in x)$ ou $\forall x \exists y (y \notin x)$.
		\item $\exists ! x \forall y (y \notin x)$.
		\item $\exists ! y (y \in x)$.
		\item $\exists x \forall y ((y \in x) \leftrightarrow y = \phi)$.
		\item $\forall z (z \in y \leftrightarrow \forall w (w \in z \to w \in x))$.
	\end{enumerate}
\end{solucao}

\begin{exercicio}
	Marque as ocorr�ncias de vari�veis livres nas f�rmulas abaixo
\end{exercicio}
\begin{enumerate}[label=(\alph{*})]
	\item $(\forall x (x=y)) \rightarrow (x \in y ) $
	\item $ \forall x ((x=y) \rightarrow (x \in y))$
	\item $\forall x(x=x) \rightarrow (\forall y \exists Z ((x=y) \land (y=z)) \rightarrow \neg(x\in y))$
	\item $ \forall x \exists y(\neg(x=y) \land \forall z ((x \in y) \leftrightarrow \forall w ((w \in z ) \rightarrow (w \in x )))) $
	\item $(x=y)\rightarrow \exists (x=y) $
\end{enumerate}

\begin{solucao}
	\begin{enumerate}[label=(\alph{*})]
    \item $x$ e $y$
    \item $y$ 
    \item $x$
    \item N�o h� vari�veis livres.
    \item $x$ e $y$
\end{enumerate}
\end{solucao}

\begin{exercicio}
	Escreva as subf�rmulas de cada f�rmula do exerc�cio 2.
\end{exercicio}

\begin{solucao}

\begin{enumerate}[label=(\alph{*})]
	\item 
	\begin{itemize}
		\item $(\forall x (x = y)) \rightarrow (x \in y)$
		\item $(\forall x (x = y))$
		\item $(x = y)$
		\item $(x \in y)$
	\end{itemize}
	
	\item 
	\begin{itemize}
		\item $\forall x ((x = y) \rightarrow (x \in y))$
		\item $(x = y) \rightarrow (x \in y)$
		\item $(x = y)$
		\item $(x \in y)$
	\end{itemize}
	
	\item 
	\begin{itemize}
		\item $\forall x (x = x) \rightarrow (\forall y \exists z (((x = y) \land (y = z)) \rightarrow \neg (x \in y)))$
		\item $\forall x (x = x)$
		\item $(x = x)$
		\item $\forall z \exists y (((x = y) \land (y = z)) \rightarrow \neg (x \in y))$
		\item $((x = y) \land (y = z))$
		\item $(x = y)$
		\item $(y = z)$
		\item $\neg (x \in y)$
		\item $(x \in y)$
	\end{itemize}
	
	\item 
	\begin{itemize}
		\item $(x = y) \rightarrow \exists y (x = y)$
		\item $(x = y)$
		\item $\exists y (x = y)$
	\end{itemize}
\end{enumerate}

\end{solucao}
	\chapter{Primeiros Axiomas}

\setcounter{ex}{0}

\begin{exercicio}
	Usando o axioma da extens�o, prove que $\{\emptyset\}$ e $\{\{\emptyset\}\}$ s�o conjuntos diferentes.
\end{exercicio}
\begin{solucao}
	O conjunto $\{\emptyset\}$ tem como �nico elemento $\emptyset$, enquanto $\{\{\emptyset\}\}$ tem como �nico elemento $\{\emptyset\}$. Como $\emptyset\neq\{\emptyset\}$, segue do axioma da extens�o que $\{\emptyset\}$ e $\{\{\emptyset\}\}$ s�o conjuntos distintos.
\end{solucao}

\begin{exercicio}
	Para cada par de conjuntos abaixo, decida qual(is) dos s�mbolos $\in$ e $\subset$ torna(m) a f�rmula verdadeira (assumindo que esses conjuntos existem). Lembre-se de que a resposta tamb�m pode ser ambos os s�mbolos ou nenhum deles. Justifique cada resposta.
	\begin{enumerate}[label=(\alph{*})]
		\item $\{\emptyset\}\ldots\{\emptyset,\{\emptyset\}\}$
		\item $\{\emptyset\}\ldots\{\{\emptyset\}\}$
		\item $\{1,2,3\}\ldots\{\{1\},\{2\},\{3\}\}$
		\item $\{1,2,3\}\ldots\{\{1\},\{1,2\},\{1,2,3\}\}$
		\item $\{1,2\}\ldots\{1,\{1\},2,\{2\},\{3\}\}$
		\item $\{\{1\},\{2\}\}\ldots\{\{1,2\}\}$
	\end{enumerate}
\end{exercicio}

\begin{enumerate}[label=(\alph{*})]
	\item $\{\emptyset\}\in\{\emptyset,\{\emptyset\}\}$ e $\{\emptyset\}\subset\{\emptyset,\{\emptyset\}\}$.
	\item $\{\emptyset\}\in\{\{\emptyset\}\}$.
	\item N�o vale $\in$ nem $\subset$.
	\item $\{1,2,3\}\in\{\{1\},\{1,2\},\{1,2,3\}\}$.
	\item $\{1,2\}\subset\{1,\{1\},2,\{2\},\{3\}\}$.
	\item N�o vale $\in$ nem $\subset$.
\end{enumerate}


\begin{exercicio}
	Seja $x$ o conjunto $\{\emptyset,\{\emptyset\},\emptyset,\{\emptyset,\{\emptyset\}\}\}$.
	\begin{enumerate}[label=(\alph{*})]
		\item Quantos elementos tem o conjunto $x$?
		\item Descreva todos os subconjuntos de $x$.
		\item Descreva, usando chaves e v�rgula, o conjunto de todos os subconjuntos de $x$.
		\item Quantos elementos o conjunto dos subconjuntos de $x$ possui?
		\item Prove que o conjunto $x$ existe.
	\end{enumerate}
\end{exercicio}

\begin{enumerate}[label=(\alph{*})]
	\item 3 elementos.
	\item $\emptyset,\{\emptyset\},\{\{\emptyset,\{\emptyset\}\}\},\{\emptyset,\{\emptyset\}\},\{\{\emptyset,\{\emptyset,\{\emptyset\}\}\}\},\{\emptyset,\{\emptyset,\{\emptyset\}\}\},\{\emptyset,\{\emptyset\},\{\emptyset,\{\emptyset\}\}\}\}$.
	\item $\{\emptyset,\{\emptyset\},\{\{\emptyset,\{\emptyset\}\}\},\{\emptyset,\{\emptyset\}\},\{\{\emptyset,\{\emptyset,\{\emptyset\}\}\}\},\{\emptyset,\{\emptyset,\{\emptyset\}\}\},\{\emptyset,\{\emptyset\},\{\emptyset,\{\emptyset\}\}\}\}\}$.
	\item 8 elementos.
	\item 
	\begin{description}
		\item[Solu��o 1:] Pelo axioma do vazio, existe $\emptyset$. Pelo axioma das partes, existe $\mathcal{P}(\emptyset)=\{\emptyset,\{\emptyset\}\}$, e ent�o $x=\mathcal{P}(\emptyset)\cup\{\mathcal{P}(\emptyset)\}=\{\emptyset,\{\emptyset\},\{\emptyset,\{\emptyset\}\}\}$ pelo Teorema 3.7. $\qed$
		\item[Solu��o 2:] Peguemos do conjunto $\omega$, da defini��o 3.19, o elemento que � representante do n�mero tr�s, ou seja, $y = \{\emptyset, \{\emptyset\}, \{\emptyset, \{\emptyset\}\}\} \in \omega$. Pelo Axioma da Extens�o, $x = y$. $\qed$ 
	\end{description}
\end{enumerate}

\begin{exercicio}
	Prove que para todos conjuntos $x$ e $y$
	\begin{enumerate}[label=(\alph{*})]
		\item $x\subset x$;
		\item $x\in y$ se, e somente se, $\{x\}\subset y$;
		\item $\bigcup\mathcal{P}(x)=x$;
		\item se $x\subset y$, ent�o $\bigcup x\subset \bigcup y$.
	\end{enumerate}
\end{exercicio}

\begin{enumerate}[label=(\alph{*})]
	\item 
		\begin{solucao} 
			Se $x$ n�o � vazio, todo elemento de $x$ � elemento de $x$, logo $x\subset x$. Se $x$ � vazio, como $\emptyset$ est� contido em qualquer conjunto pelo Teorema 3.4, temos que $\emptyset\subset \emptyset$. Em qualquer caso, $x\subset x$. 
		\end{solucao}
	\item 
		\begin{solucao}
			Se $x\in y$, como $x$ � o �nico elemento do conjunto $\{x\}$, temos que $\{x\}\subset y$. Reciprocamente, se $\{x\}\subset y$, ent�o $x\in y$.
		\end{solucao}
	\item 
		\begin{solucao}
			Dado $u\in\bigcup\mathcal{P}(x)$, existe $v\in\mathcal{P}(x)$ tal que $u\in v$. Como $\mathcal{P}(x)$ � o conjunto de todos os subconjuntos de $x$, $v\subset x$, logo $u\in x$. Portanto $\bigcup\mathcal{P}(x)\subset x$.
			
			Reciprocamente, como $x\subset x$ pelo item (a), ent�o $x\in\mathcal{P}(x)$, logo qualquer elemento $y\in x$ ser� elemento de $\bigcup\mathcal{P}(x)$, e assim $x\subset \bigcup\mathcal{P}(x)$.
			
			Pelo axioma da extens�o, $\bigcup\mathcal{P}(x)=x$.
		\end{solucao}
	\item 
		\begin{solucao}
			Dado $u\in\bigcup x$, existe $v\in x$ tal que $u\in v$. Como $x\subset y$, $v\in y$, logo $u\in\bigcup y$ e, portanto, $\bigcup x\subset\bigcup y$.
		\end{solucao}
\end{enumerate}

\begin{exercicio}
	Escreva uma f�rmula de primeira ordem, na linguagem da teoria dos conjuntos, com quatro vari�veis livres, que represente o conjunto $\{x,y,z\}$.
\end{exercicio}
\begin{solucao}
	$\forall w((w\in u)\leftrightarrow((w=x)\vee(w=y)\vee(w=z)))$.
\end{solucao}

\begin{exercicio}
	Escreva os seguintes conjuntos, listando seus elementos entre chaves:
	\begin{enumerate}[label=(\alph{*})]
		\item $\bigcup\{\{0,1\},\{\{1\}\},\{1,2\},\{\{1,2\}\}\}$;
		\item $\mathcal{P}(\{\emptyset,\{\emptyset\}\})$.
	\end{enumerate}
\end{exercicio}

\begin{enumerate}[label=(\alph{*})]
	\item $\{0,1,\{1\},2,\{1,2\}\}$.
	\item $\{\emptyset,\{\emptyset\},\{\{\emptyset\}\},\{\emptyset,\{\emptyset\}\}\}$.
\end{enumerate}


\begin{exercicio}
	Prove que n�o existe o conjunto de todos os conjuntos unit�rios.
	
	\emph{Dica:} Assuma, por absurdo, a exist�ncia do conjunto de todos os conjuntos unit�rios e prove a exist�ncia do conjunto de todos os conjuntos.
\end{exercicio}

\begin{proof}[\textbf{Solu��o 1}:]
	Suponha que existe $x$ tal que $\forall y(\{y\}\in x)$. Ent�o $\{x\}\in x$. Como $x\in\{x\}$, temos uma contradi��o com o Teorema 3.14. Portanto $\nexists x\forall y(\{y\}\in x)$.
\end{proof}
	
\begin{proof}[\textbf{Solu��o 2}:]
	Suponha, por absurdo, que $u$ seja o conjunto de todos os conjuntos unit�rios. Seja $x$ um conjunto arbitr�rio, ent�o $\{x\} \in u$ por hip�tese e, pelo Axioma da Uni�o, teremos $x \in \bigcup x$. Isso faz de $\bigcup x$ o conjunto de todos os conjuntos, o que � um absurdo pelo Teorema 3.10.
\end{proof}
	

\begin{exercicio}
	Prove que para todo conjunto $X$ existe o conjunto
	$$\{\{x\}:x\in X\}$$
\end{exercicio}
\begin{solucao}
	Pelo axioma das partes, existe o conjunto $\mathcal{P}(X)$. Para cada $x\in X$, temos que $\{x\}\subset X$, logo $\{x\}\in\mathcal{P}(X)$. Pelo axioma da separa��o, existe o conjunto $\{y\in \mathcal{P}(X):\exists x((x\in X)\wedge(\{x\}=y))\}$, que, via axioma da extens�o, � o conjunto procurado.
\end{solucao}

\begin{exercicio}
	Sendo $x$ um conjunto n�o vazio, prove que
	\begin{enumerate}[label=(\alph{*})]
		\item $\forall y(y\in x\to(\bigcap x\subset y))$;
		\item $x\subset y\to \bigcap y\subset \bigcap x$.
	\end{enumerate}
\end{exercicio}

\begin{enumerate}[label=(\alph{*})]
	\item
		\begin{description}
			\item[Solu��o 1] Dados $y\in x$ e $z\in\bigcap x$, temos que $\forall w((w\in x)\to(z\in w))$. Em particular, $z\in y$, logo $\bigcap x\subset y$. $\qed$
			\item[Solu��o 2] Seja $x$ um conjunto qualquer n�o vazio e $y$ um de seus elementos. Se $\bigcap x = \emptyset$ ent�o $\bigcap x \subset y$ pelo Teorema 3.4. Caso contr�rio, escolha arbitrariamente um elemento $z$ de $\bigcap x$, ent�o $z$ pertence a todos os elementos de $x$, desse modo, teremos $z \in y$ e, novamente, $\bigcap x \subset y$. $\qed$
		\end{description}
	\item Dado $z\in\bigcap y$, temos que $z$ � elemento de qualquer $w\in y$. Como $x\subset y$, os elementos de $x$ s�o tamb�m elementos de  $y$, logo $z$ � elemento de qualquer $w\in x$ em particular, portanto $z\in\bigcap x$. Desta forma $\bigcap y\subset\bigcap x$.
\end{enumerate}


\begin{exercicio}
	Escreva na linguagem da l�gica de primeira ordem, sem abreviaturas, a seguinte f�rmula:
	$$x\in\bigcup\bigcap(y\cup(w\backslash z)).$$
\end{exercicio}
\begin{solucao}
	Faremos por partes. Dizer que $x\in\bigcup a$ para algum conjunto $a$ significa que $\exists u((x\in u)\wedge(u\in a))$. Chame $a=\bigcap b$, sendo $b$ um conjunto n�o vazio. Ent�o $x\in a\leftrightarrow\forall v((v\in b)\rightarrow(x\in v))$. Desta forma, $x\in \bigcup\bigcap b$ significa
	$$\exists u((x\in u)\wedge\forall v((v\in b)\rightarrow(u\in v))).$$
	Fa�a $b=y\cup c$. Ent�o $x\in c\leftrightarrow(x\in y)\vee(x\in c)$. Assim
	$$x\in \bigcup\bigcap (y\cup c)\leftrightarrow\exists u((x\in u)\wedge\forall v(((v\in y)\vee(v\in c))\rightarrow(u\in v))).$$
	Por fim, fa�a $c=w\backslash z$. Ent�o $x\in c\leftrightarrow(x\in w)\wedge\neg(x\in z)$. Portanto $x\in \bigcup\bigcap (y\cup(w\backslash z))$ significa
	$$\exists u((x\in u)\wedge\forall v(((v\in y)\vee((v\in w)\wedge\neg(v\in z)))\rightarrow(u\in v))).$$
\end{solucao}

\begin{exercicio}
	Usando o axioma da regularidade, prove que:
	\begin{enumerate}[label=(\alph{*})]
		\item n�o existem $x,y,z$ tais que $x\in y,y\in z$ e $z\in x$;
		\item n�o existem $w,x,y,z$ tais que $w\in x,x\in y,y\in z$ e $z\in w$.
	\end{enumerate}
\end{exercicio}
\begin{solucao}
	\begin{enumerate}[label=(\alph{*})]
		\item Pelo axioma do par, existem $\{x,y\}$ e $\{y,z\}$. Tome $w=\{x,y\}\bigcup \{y,z\}=\{x,y,z\}$. Pelo axioma da regularidade, existe $u\in w$ tal que $u\cap w=\emptyset$. Se for $u=x$, ent�o $y\notin x$ e $z\notin x$. Se for $u=y$, ent�o $x\notin y$ e $z\notin y$. Se for $u=z$, ent�o $x\notin z$ e $y\notin z$. Em qualquer caso, n�o podemos ter $x\in y$, $y\in z$ e $z\in x$ simultaneamente.
		\item Basta repetir o argumento acima para o conjunto $\{w,x,y,z\}$.
	\end{enumerate}
\end{solucao}

\begin{exercicio}
	Prove que n�o existe $x$ tal que $\mathcal{P}(x)=x$.
\end{exercicio}
\begin{solucao}[ 1]
	Se $\mathcal{P}(x)=x$, como $x\subset x$, ent�o $x\in \mathcal{P}(x)$, contradizendo o Corol�rio 3.15.
\end{solucao}

\begin{solucao}[ 2]
	Suponha que $x$ seja um conjunto tal que $\mathcal{P}(x)=x$. Seja $y$ um elemento qualquer de $x$, ent�o tamb�m teremos $y \subset x$, dado que assumimos $\mathcal{P}(x)=x$, e portanto $x \cap y \neq \emptyset$, o que � um absurdo pelo Axioma da Regularidade.
\end{solucao}

\begin{exercicio}
	Escreva o conjunto $\mathcal{P}(3\backslash 1)$, utilizando apenas os seguintes s��mbolos: as chaves, a v��rgula e o s��mbolo de conjunto vazio.
\end{exercicio}
\begin{solucao}
	$\mathcal{P}(3\backslash 1)=\{\emptyset,\{\emptyset\},\{\{\emptyset,\{\emptyset\}\}\},\{\emptyset,\{\emptyset,\{\emptyset\}\}\}\}$.
\end{solucao}

\begin{exercicio}
	Prove, a partir dos axiomas de Peano, os seguintes teoremas:
	\begin{enumerate}[label=(\alph{*})]
		\item Todo n�mero natural � diferente do seu sucessor.
		\item Zero � o �nico n�mero natural que n�o � sucessor de nenhum n�mero natural.
	\end{enumerate}
\end{exercicio}

\begin{enumerate}[label=(\alph{*})]
	\item
	\begin{solucao}[ 1]
		Tese: $\forall n((n\in\omega)\wedge(n\neq n^+))$.
		
		Para $n=0$, $n^+\neq0$, pois $0$ n�o � sucessor de nenhum n�mero natural, logo a tese � verdadeira para $n=0$.
		
		Suponha que a tese � verdadeira para algum $n\in\omega$. Pelo axioma 3, $n$ e $n^+$ devem ter sucessores distintos, isto �, $n^+\neq(n^+)^+$. Isto mostra que a tese � verdadeira para $n^+$. Pelo axioma 5, a tese � verdadeira para todo $n$ natural.
	\end{solucao}
	
	\begin{solucao}[ 2]
			Usemos $\omega$ da defini��o 3.19 como modelo para $\mathbb{N}$. Pelo Teorema 3.20, $\omega$ satisfaz os Axiomas de Peano e, portanto, podemos aplicar o Princ��pio da Indu��o Finita. Por indu��o, verifiquemos que:
			\begin{description}
				\item[\textbf{Caso Base}:] Pelo Axioma da Extens�o temos que $\emptyset \neq (\emptyset \cup \{\emptyset\}) = \{\emptyset\} = \emptyset^+$.
				\item[\textbf{Passo indutivo}:] Seja $x$ um elemento qualquer de $\omega$ e suponha que $x \neq x^+$, ent�o: \newline \newline $(x^+)^+ = x^+ \cup \{x^+\}$ \newline $=  (x \cup \{x\}) \cup \{x^+\}$ \newline $= x \cup (\{x\} \cup \{x^+\})$ \newline $= x \cup \{x, x^+\} \neq x \cup \{x\} = x^+$ \newline \newline Com isso garantimos, pelo quinto Axioma de Peano, que todo n�mero natural � diferente de seu sucessor.
			\end{description}
		\end{solucao}
	
	\item  
	\begin{solucao}[ 1]
		
		\textbf{Tese:}
			\[
			\forall n \big((n \in \omega) \wedge \nexists m \big((m \in \omega) \wedge (m^+ = n)\big) \to (n = 0)\big).
			\]
			
			\textbf{Contrapositiva:}  
			\[
			\forall n \big((n \neq 0) \to ((n \notin \omega) \vee \exists m \big((m \in \omega) \wedge (m^+ = n)\big))\big).
			\]
			
			Demonstraremos a contrapositiva, por ser equivalente � tese. Para isso, precisaremos da seguinte proposi��o:
			
			\vspace{0.3cm}
			\textbf{Proposi��o:} Seja $\sigma$ um subconjunto n�o vazio de $\omega$. Se $0 \in \sigma$ e, para cada $n \in \sigma$, valer $n^+ \in \sigma$, ent�o $\sigma = \omega$.
			
			\vspace{0.3cm}
			\textbf{Prova da Proposi��o:} Aplicando o axioma 5 � f�rmula $P(\sigma) = (x \in \sigma)$, obtemos $\omega \subset \sigma$. Como $\sigma \subset \omega$, ent�o $\sigma = \omega$.  
			
			Seja 
			\[
			\sigma = \{0\} \cup \{n \in \omega : \exists m ((m \in \omega) \wedge (m^+ = n))\}.
			\]
			Temos que $0 \in \sigma$ e, para cada $n \in \sigma$, $n^+ \in \sigma$. Pela proposi��o acima, $\sigma = \omega$.  
			
			Pelo axioma 4, 
			\[
			\{0\} \cap \{n \in \omega : \exists m ((m \in \omega) \wedge (m^+ = n))\} = \emptyset.
			\]
			Portanto, para qualquer $n \in \omega$ com $n \neq 0$,  
			\[
			\exists m ((m \in \omega) \wedge (m^+ = n)).
			\]
			Isto demonstra, por contraposi��o, a tese.
	\end{solucao}
	
	\begin{solucao}[ 2]
		Seja $n$ um elemento de $\omega$ tal que $n \neq \emptyset$ e suponha que $n$ n�o � sucessor de nenhum outro n�mero natural de $\omega$. Ent�o, pelo Teorema 3.21, item $e$, temos os seguintes cen�rios:
		
		\begin{enumerate}[label=\textit{\arabic*� --}, left=0pt, itemsep=0.5em]
			\item $\emptyset = n$: o que contradiz nossa hip�tese.
			\item $\emptyset \in n$: ent�o, pelo item $b$ do Teorema 3.21, teremos os seguintes subcasos:
			\begin{itemize}
				\item $\emptyset^+ = n$: o que novamente contradiz nossa hip�tese.
				\item $\emptyset^+ \in n$: ent�o $n$ existe gra�as a sucessivas aplica��es\footnote{A ideia de aplicar uma mesma propriedade de forma sucessiva ser� formalizada no Cap�tulo 4 com o Teorema da Recurs�o.} da \textbf{Defini��o 3.16}, e h� um 
				\[
				m = (\cdots((\emptyset^+)^+)^+\cdots)^+
				\]
				tal que $m^+ = n$, contrariando nossa hip�tese.
			\end{itemize}
			\item $n \in \emptyset$: o que � um absurdo.
		\end{enumerate}
		
		Como esgotamos todos os cen�rios e, em todos eles, chegamos a uma contradi��o, n�o pode haver $n$, al�m de $\emptyset$, em $\omega$ que n�o seja sucessor de nenhum outro n�mero natural.
	\end{solucao}
\end{enumerate}

\begin{exercicio}
	Prove que:
	\begin{enumerate}[label=(\alph{*})]
		\item para todo $n\in\omega$, $\emptyset\in n$ ou $\emptyset=n$;
		\item para todos $n,m\in\omega$, se $m\in n$, ent�o $m\subset n$.
	\end{enumerate}
\end{exercicio}

\begin{enumerate}[label=(\alph{*})]
	\item Para $n=0$, $n=\emptyset$, logo a tese � verdadeira.
	
	Suponha ser verdade para $n$. Se $\emptyset=n$, ent�o $\emptyset\in n^+=n\cup\{n\}$. Se $\emptyset\in n$, ent�o $\emptyset\in n^+$, e a tese � verdadeira para $n^+$
	
	Pelo axioma 5, $\emptyset\in n$ ou $\emptyset=n$ para qualquer $n\in\omega$.
	\item J� feito na prova do Teorema 3.21, item (c).
\end{enumerate}


\begin{exercicio}
	A uni�o de dois conjuntos indutivos � necessariamente um conjunto indutivo? Justifique sua resposta.
\end{exercicio}
\begin{solucao}
	
	\underline{Resposta:} Sim.
	
	\underline{Justificativa:} Sejam $A$ e $B$ dois conjuntos indutivos. Como $\emptyset\in A$ e $\emptyset\in B$, ent�o $\emptyset\in A\cup B$. Se $x\in A\cup B$, ent�o $x\in A$ ou $x\in B$, logo $x^+\in A$ ou $x^+\in B$, o que implica $x^+\in A\cup B$. Isto mostra que $A\cup B$ � indutivo.
\end{solucao}
	\include{capitulos/cap4}
	\include{capitulos/cap5}
	\include{capitulos/cap6}
	\include{capitulos/cap7}
	\include{capitulos/cap8}
	\include{capitulos/cap9}
	\include{capitulos/cap10}
	\include{capitulos/cap11}
	\include{capitulos/cap12}
	
	%\bibliographystyle{amsplain}
	%\printbibliography[title=Refer�ncias]
	%\bibliography{bibliografia.bib}
	\begin{thebibliography}{10}
		\bibitem[Fa]{Fajardo} FAJARDO, R. A. dos S. \textit{A Teoria dos Conjuntos e os Fundamentos da Matem�tica}. S�o Paulo: Edusp, 2024.
		\bibitem[Li]{Lima16} LIMA, E. L. \textit{Curso de an�lise}. 14 ed. Rio de Janeiro: Impa, 2016. v. 1.
	\end{thebibliography}
\end{document}