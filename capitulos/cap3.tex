\chapter{Primeiros Axiomas}

\setcounter{ex}{0}

\begin{exercicio}
	Usando o axioma da extens�o, prove que $\{\emptyset\}$ e $\{\{\emptyset\}\}$ s�o conjuntos diferentes.
\end{exercicio}
\begin{solucao}
	content
\end{solucao}

\begin{exercicio}
	Para cada par de conjuntos abaixo, decida qual(is) dos s�mbolos $\in$ e $\subset$ torna(m) a f�rmula verdadeira (assumindo que esses conjuntos existem). Lembre-se de que a resposta tamb�m pode ser ambos os s�mbolos ou nenhum deles. Justifique cada resposta.
	\begin{enumerate}[label=(\alph{*})]
		\item $\{\emptyset\}\ldots\{\emptyset,\{\emptyset\}\}$
		\item $\{\emptyset\}\ldots\{\{\emptyset\}\}$
		\item $\{1,2,3\}\ldots\{\{1\},\{2\},\{3\}\}$
		\item $\{1,2,3\}\ldots\{\{1\},\{1,2\},\{1,2,3\}\}$
		\item $\{1,2\}\ldots\{1,\{1\},2,\{2\},\{3\}\}$
		\item $\{\{1\},\{2\}\}\ldots\{\{1,2\}\}$
	\end{enumerate}
\end{exercicio}
\begin{solucao}
	content
\end{solucao}

\begin{exercicio}
	Seja $x$ o conjunto $\{\emptyset,\{\emptyset\},\emptyset,\{\emptyset,\{\emptyset\}\}\}$.
	\begin{enumerate}[label=(\alph{*})]
		\item Quantos elementos tem o conjunto $x$?
		\item Descreva todos os subconjuntos de $x$.
		\item Descreva, usando chaves e v�rgula, o conjunto de todos os subconjuntos de $x$.
		\item Quantos elementos o conjunto dos subconjuntos de $x$ possui?
		\item Prove que o conjunto $x$ existe.
	\end{enumerate}
\end{exercicio}
\begin{solucao}
	content
\end{solucao}

\begin{exercicio}
	Prove que para todos conjuntos $x$ e $y$
	\begin{enumerate}[label=(\alph{*})]
		\item $x\subset x$;
		\item $x\in y$ se, e somente se, $\{x\}\subset y$;
		\item $\bigcup\mathcal{P}(x)=x$;
		\item se $x\subset y$, ent�o $\bigcup x\subset \bigcup y$.
	\end{enumerate}
\end{exercicio}
\begin{solucao}
	content
\end{solucao}

\begin{exercicio}
	Escreva uma f�rmula de primeira ordem, na linguagem da teoria dos conjuntos, com quatro vari�veis livres, que represente o conjunto $\{x,y,z\}$.
\end{exercicio}
\begin{solucao}
	content
\end{solucao}

\begin{exercicio}
	Escreva os seguintes conjuntos, listando seus elementos entre chaves:
	\begin{enumerate}[label=(\alph{*})]
		\item $\bigcup\{\{0,1\},\{\{1\}\},\{1,2\},\{\{1,2\}\}\}$;
		\item $\mathcal{P}(\{\emptyset,\{\emptyset\}\})$.
	\end{enumerate}
\end{exercicio}
\begin{solucao}
	content
\end{solucao}

\begin{exercicio}
	Prove que n�o existe o conjunto de todos os conjuntos unit�rios.
	
	\emph{Dica:} Assuma, por absurdo, a exist�ncia do conjunto de todos os conjuntos unit�rios e prove a exist�ncia do conjunto de todos os conjuntos.
\end{exercicio}
\begin{solucao}
	content
\end{solucao}

\begin{exercicio}
	Prove que para todo conjunto $X$ existe o conjunto
	$$\{\{x\}:x\in X\}$$
\end{exercicio}
\begin{solucao}
	content
\end{solucao}

\begin{exercicio}
	Senso $x$ um conjunto n�o vazio, prove que
	\begin{enumerate}[label=(\alph{*})]
		\item $\forall y(y\in x\to(\bigcap x\subset y))$;
		\item $x\subset y\to \bigcap y\subset \bigcap x$.
	\end{enumerate}
\end{exercicio}
\begin{solucao}
	content
\end{solucao}

\begin{exercicio}
	Escreva na linguagem da l�gica de primeira ordem, sem abreviaturas, a seguinte f�rmula:
	$$x\in\bigcup\bigcap(y\cup(w\backslash z)).$$
\end{exercicio}
\begin{solucao}
	content
\end{solucao}

\begin{exercicio}
	Usando o axioma da regularidade, prove que:
	\begin{enumerate}[label=(\alph{*})]
		\item n�o existem $x,y,z$ tais que $x\in y,y\in z$ e $z\in x$;
		\item n�o existem $w,x,y,z$ tais que $w\in x,x\in y,y\in z$ e $z\in w$.
	\end{enumerate}
\end{exercicio}
\begin{solucao}
	content
\end{solucao}

\begin{exercicio}
	Prove que n�o existe $x$ tal que $\mathcal{P}(x)=x$.
\end{exercicio}
\begin{solucao}
	content
\end{solucao}

\begin{exercicio}
	Escreva o conjunto $\mathcal{P}(3\backslash 1)$, utilizando apenas os seguintes s�mbolos: as chaves, a v�rgula e o s�mbolo de conjunto vazio.
\end{exercicio}
\begin{solucao}
	content
\end{solucao}

\begin{exercicio}
	Prove, a partir dos axiomas de Peano, os seguintes teoremas:
	\begin{enumerate}[label=(\alph{*})]
		\item Todo n�mero natural � diferente do seu sucessor.
		\item Zero � o �nico n�mero natural que n�o � sucessor de nenhum n�mero natural.
	\end{enumerate}
\end{exercicio}
\begin{solucao}
	content
\end{solucao}

\begin{exercicio}
	Prove que:
	\begin{enumerate}[label=(\alph{*})]
		\item para todo $n\in\omega$, $\emptyset\in n$ ou $\emptyset=n$;
		\item para todos $n,m\in\omega$, se $m\in n$, ent�o $m\subset n$.
	\end{enumerate}
\end{exercicio}
\begin{solucao}
	content
\end{solucao}

\begin{exercicio}
	A uni�o de dois conjuntos indutivos � necessariamente um conjunto indutivo? Justifique sua resposta.
\end{exercicio}
\begin{solucao}
	content
\end{solucao}