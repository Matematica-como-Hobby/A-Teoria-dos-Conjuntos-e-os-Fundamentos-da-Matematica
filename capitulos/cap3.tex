\chapter{Primeiros Axiomas}

\setcounter{ex}{0}

\begin{exercicio}
	Usando o axioma da extensão, prove que $\{\emptyset\}$ e $\{\{\emptyset\}\}$ são conjuntos diferentes.
\end{exercicio}
\begin{solucao}
	O conjunto $\{\emptyset\}$ tem como único elemento $\emptyset$, enquanto $\{\{\emptyset\}\}$ tem como único elemento $\{\emptyset\}$. Como $\emptyset\neq\{\emptyset\}$, segue do axioma da extensão que $\{\emptyset\}$ e $\{\{\emptyset\}\}$ são conjuntos distintos.
\end{solucao}

\begin{exercicio}
	Para cada par de conjuntos abaixo, decida qual(is) dos símbolos $\in$ e $\subset$ torna(m) a fórmula verdadeira (assumindo que esses conjuntos existem). Lembre-se de que a resposta também pode ser ambos os símbolos ou nenhum deles. Justifique cada resposta.
	\begin{enumerate}[label=(\alph{*})]
		\item $\{\emptyset\}\ldots\{\emptyset,\{\emptyset\}\}$
		\item $\{\emptyset\}\ldots\{\{\emptyset\}\}$
		\item $\{1,2,3\}\ldots\{\{1\},\{2\},\{3\}\}$
		\item $\{1,2,3\}\ldots\{\{1\},\{1,2\},\{1,2,3\}\}$
		\item $\{1,2\}\ldots\{1,\{1\},2,\{2\},\{3\}\}$
		\item $\{\{1\},\{2\}\}\ldots\{\{1,2\}\}$
	\end{enumerate}
\end{exercicio}

\begin{enumerate}[label=(\alph{*})]
	\item $\{\emptyset\}\in\{\emptyset,\{\emptyset\}\}$ e $\{\emptyset\}\subset\{\emptyset,\{\emptyset\}\}$.
	\item $\{\emptyset\}\in\{\{\emptyset\}\}$.
	\item Não vale $\in$ nem $\subset$.
	\item $\{1,2,3\}\in\{\{1\},\{1,2\},\{1,2,3\}\}$.
	\item $\{1,2\}\subset\{1,\{1\},2,\{2\},\{3\}\}$.
	\item Não vale $\in$ nem $\subset$.
\end{enumerate}


\begin{exercicio}
	Seja $x$ o conjunto $\{\emptyset,\{\emptyset\},\emptyset,\{\emptyset,\{\emptyset\}\}\}$.
	\begin{enumerate}[label=(\alph{*})]
		\item Quantos elementos tem o conjunto $x$?
		\item Descreva todos os subconjuntos de $x$.
		\item Descreva, usando chaves e vírgula, o conjunto de todos os subconjuntos de $x$.
		\item Quantos elementos o conjunto dos subconjuntos de $x$ possui?
		\item Prove que o conjunto $x$ existe.
	\end{enumerate}
\end{exercicio}

\begin{enumerate}[label=(\alph{*})]
	\item 3 elementos.
	\item $\emptyset,\{\emptyset\},\{\{\emptyset,\{\emptyset\}\}\},\{\emptyset,\{\emptyset\}\},\{\{\emptyset,\{\emptyset,\{\emptyset\}\}\}\},\{\emptyset,\{\emptyset,\{\emptyset\}\}\},\{\emptyset,\{\emptyset\},\{\emptyset,\{\emptyset\}\}\}\}$.
	\item $\{\emptyset,\{\emptyset\},\{\{\emptyset,\{\emptyset\}\}\},\{\emptyset,\{\emptyset\}\},\{\{\emptyset,\{\emptyset,\{\emptyset\}\}\}\},\{\emptyset,\{\emptyset,\{\emptyset\}\}\},\{\emptyset,\{\emptyset\},\{\emptyset,\{\emptyset\}\}\}\}\}$.
	\item 8 elementos.
	\item 
	\begin{description}
		\item[Solução por Allan Kenedy:] Pelo axioma do vazio, existe $\emptyset$. Pelo axioma das partes, existe $\mathcal{P}(\emptyset)=\{\emptyset,\{\emptyset\}\}$, e então $x=\mathcal{P}(\emptyset)\cup\{\mathcal{P}(\emptyset)\}=\{\emptyset,\{\emptyset\},\{\emptyset,\{\emptyset\}\}\}$ pelo Teorema 3.7. $\qed$
		\item[Solução por Cloves Paiva:] Peguemos do conjunto $\omega$, da definição 3.19, o elemento que é representante do número três, ou seja, $y = \{\emptyset, \{\emptyset\}, \{\emptyset, \{\emptyset\}\}\} \in \omega$. Pelo Axioma da Extensão, $x = y$. $\qed$ 
	\end{description}
\end{enumerate}

\begin{exercicio}
	Prove que para todos conjuntos $x$ e $y$
	\begin{enumerate}[label=(\alph{*})]
		\item $x\subset x$;
		\item $x\in y$ se, e somente se, $\{x\}\subset y$;
		\item $\bigcup\mathcal{P}(x)=x$;
		\item se $x\subset y$, então $\bigcup x\subset \bigcup y$.
	\end{enumerate}
\end{exercicio}

\begin{enumerate}[label=(\alph{*})]
	\item 
		\begin{solucao} 
			Se $x$ não é vazio, todo elemento de $x$ é elemento de $x$, logo $x\subset x$. Se $x$ é vazio, como $\emptyset$ está contido em qualquer conjunto pelo Teorema 3.4, temos que $\emptyset\subset \emptyset$. Em qualquer caso, $x\subset x$. 
		\end{solucao}
	\item 
		\begin{solucao}
			Se $x\in y$, como $x$ é o único elemento do conjunto $\{x\}$, temos que $\{x\}\subset y$. Reciprocamente, se $\{x\}\subset y$, então $x\in y$.
		\end{solucao}
	\item 
		\begin{solucao}
			Dado $u\in\bigcup\mathcal{P}(x)$, existe $v\in\mathcal{P}(x)$ tal que $u\in v$. Como $\mathcal{P}(x)$ é o conjunto de todos os subconjuntos de $x$, $v\subset x$, logo $u\in x$. Portanto $\bigcup\mathcal{P}(x)\subset x$.
			
			Reciprocamente, como $x\subset x$ pelo item (a), então $x\in\mathcal{P}(x)$, logo qualquer elemento $y\in x$ será elemento de $\bigcup\mathcal{P}(x)$, e assim $x\subset \bigcup\mathcal{P}(x)$.
			
			Pelo axioma da extensão, $\bigcup\mathcal{P}(x)=x$.
		\end{solucao}
	\item 
		\begin{solucao}
			Dado $u\in\bigcup x$, existe $v\in x$ tal que $u\in v$. Como $x\subset y$, $v\in y$, logo $u\in\bigcup y$ e, portanto, $\bigcup x\subset\bigcup y$.
		\end{solucao}
\end{enumerate}


\begin{exercicio}
	Escreva uma fórmula de primeira ordem, na linguagem da teoria dos conjuntos, com quatro variáveis livres, que represente o conjunto $\{x,y,z\}$.
\end{exercicio}
\begin{solucao}
	$\forall w ((w \in u) \implies ((w=x) \vee (w=y) \vee (w=z)))$
\end{solucao}

\begin{exercicio}
	Escreva os seguintes conjuntos, listando seus elementos entre chaves:
	\begin{enumerate}[label=(\alph{*})]
		\item $\bigcup\{\{0,1\},\{\{1\}\},\{1,2\},\{\{1,2\}\}\}$;
		\item $\mathcal{P}(\{\emptyset,\{\emptyset\}\})$.
	\end{enumerate}
\end{exercicio}

\begin{enumerate}[label=(\alph{*})]
	\item $\{0,1,\{1\},2,\{1,2\}\}$.
	\item $\{\emptyset,\{\emptyset\},\{\{\emptyset\}\},\{\emptyset,\{\emptyset\}\}\}$.
\end{enumerate}


\begin{exercicio}
	Prove que não existe o conjunto de todos os conjuntos unitários.
	
	\emph{Dica:} Assuma, por absurdo, a existência do conjunto de todos os conjuntos unitários e prove a existência do conjunto de todos os conjuntos.
\end{exercicio}

\begin{proof}[\textbf{Solução por Allan Kenedy}:]
	Suponha que existe $x$ tal que $\forall y(\{y\}\in x)$. Então $\{x\}\in x$. Como $x\in\{x\}$, temos uma contradição com o Teorema 3.14. Portanto $\nexists x\forall y(\{y\}\in x)$.
\end{proof}
	
\begin{proof}[\textbf{Solução por Cloves Paiva}:]
	Suponha, por absurdo, que $u$ seja o conjunto de todos os conjuntos unitários. Seja $x$ um conjunto arbitrário, então $\{x\} \in u$ por hipótese e, pelo Axioma da União, teremos $x \in \bigcup x$. Isso faz de $\bigcup x$ o cojunto de todos os conjuntos, o que é um absurdo pelo Teorema 3.10.
\end{proof}
	

\begin{exercicio}
	Prove que para todo conjunto $X$ existe o conjunto
	$$\{\{x\}:x\in X\}$$
\end{exercicio}
\begin{solucao}
	Pelo axioma das partes, existe o conjunto $\mathcal{P}(X)$. Para cada $x\in X$, temos que $\{x\}\subset X$, logo $\{x\}\in\mathcal{P}(X)$. Pelo axioma da separação, existe o conjunto $\{y\in \mathcal{P}(X):\exists x((x\in X)\wedge(\{x\}=y))\}$, que, via axioma da extensão, é o conjunto procurado.
\end{solucao}

\begin{exercicio}
	Sendo $x$ um conjunto não vazio, prove que
	\begin{enumerate}[label=(\alph{*})]
		\item $\forall y(y\in x\to(\bigcap x\subset y))$;
		\item $x\subset y\to \bigcap y\subset \bigcap x$.
	\end{enumerate}
\end{exercicio}

\begin{enumerate}[label=(\alph{*})]
	\item
		\begin{description}
			\item[Solução por Allan Kenedy] Dados $y\in x$ e $z\in\bigcap x$, temos que $\forall w((w\in x)\to(z\in w))$. Em particular, $z\in y$, logo $\bigcap x\subset y$. $\qed$
			\item[Solução por Cloves Paiva] Seja $x$ um conjunto qualquer não vazio e $y$ um de seus elementos. Se $\bigcap x = \emptyset$ então $\bigcap x \subset y$ pelo Teorema 3.4. Caso contrário, escolha arbitrariamente um elemento $z$ de $\bigcap x$, então $z$ pertence a todos os elementos de $x$, desse modo, teremos $z \in y$ e, novamente, $\bigcap x \subset y$. $\qed$
		\end{description}
	\item Dado $z\in\bigcap y$, temos que $z$ é elemento de qualquer $w\in y$. Como $x\subset y$, os elementos de $x$ são também elementos de  $y$, logo $z$ é elemento de qualquer $w\in x$ em particular, portanto $z\in\bigcap x$. Desta forma $\bigcap y\subset\bigcap x$.
\end{enumerate}


\begin{exercicio}
	Escreva na linguagem da lógica de primeira ordem, sem abreviaturas, a seguinte fórmula:
	$$x\in\bigcup\bigcap(y\cup(w\backslash z)).$$
\end{exercicio}
\begin{solucao}
	content
\end{solucao}

\begin{exercicio}
	Usando o axioma da regularidade, prove que:
	\begin{enumerate}[label=(\alph{*})]
		\item não existem $x,y,z$ tais que $x\in y,y\in z$ e $z\in x$;
		\item não existem $w,x,y,z$ tais que $w\in x,x\in y,y\in z$ e $z\in w$.
	\end{enumerate}
\end{exercicio}
\begin{solucao}
	\begin{enumerate}[label=(\alph{*})]
		\item Pelo axioma do par, existem $\{x,y\}$ e $\{y,z\}$. Tome $w=\{x,y\}\bigcup \{y,z\}=\{x,y,z\}$. Pelo axioma da regularidade, existe $u\in w$ tal que $u\cap w=\emptyset$. Se for $u=x$, então $y\notin x$ e $z\notin x$. Se for $u=y$, então $x\notin y$ e $z\notin y$. Se for $u=z$, então $x\notin z$ e $y\notin z$. Em qualquer caso, não podemos ter $x\in y$, $y\in z$ e $z\in x$ simultaneamente.
		\item Basta repetir o argumento acima para o conjunto $\{w,x,y,z\}$.
	\end{enumerate}
\end{solucao}

\begin{exercicio}
	Prove que não existe $x$ tal que $\mathcal{P}(x)=x$.
\end{exercicio}
\begin{proof}[\textbf{Solução por Allan Kenedy}:]
	Se $\mathcal{P}(x)=x$, como $x\subset x$, então $x\in \mathcal{P}(x)$, contradizendo o Corolário 3.15.
\end{proof}
\begin{proof}[\textbf{Solução por Cloves Paiva}:]
	Suponha que $x$ seja um conjunto tal que $\mathcal{P}(x)=x$. Seja $y$ um elemento qualquer de $x$, então também teremos $y \subset x$, dado que assumimos $\mathcal{P}(x)=x$, e portanto $x \cap y \neq \emptyset$, o que é um absurdo pelo Axioma da Regularidade.
\end{proof}

\begin{exercicio}
	Escreva o conjunto $\mathcal{P}(3\backslash 1)$, utilizando apenas os seguintes símbolos: as chaves, a vírgula e o símbolo de conjunto vazio.
\end{exercicio}
\begin{solucao}
	$\mathcal{P}(3\backslash 1)=\{\emptyset,\{\emptyset\},\{\{\emptyset,\{\emptyset\}\}\},\{\emptyset,\{\emptyset,\{\emptyset\}\}\}\}$.
\end{solucao}

\begin{exercicio}
	Prove, a partir dos axiomas de Peano, os seguintes teoremas:
	\begin{enumerate}[label=(\alph{*})]
		\item Todo número natural é diferente do seu sucessor.
		\item Zero é o único número natural que não é sucessor de nenhum número natural.
	\end{enumerate}
\end{exercicio}
\begin{solucao}
	\begin{enumerate}[label=(\alph{*})]
		\item Tese: $\forall n((n\in\omega)\wedge(n\neq n^+))$.
		
		Para $n=0$, $n^+\neq0$, pois $0$ não é sucessor de nenhum número natural, logo a tese é verdadeira para $n=0$.
		
		Suponha que a tese é verdadeira para algum $n\in\omega$. Pelo axioma 3, $n$ e $n^+$ devem ter sucessores distintos, isto é, $n^+\neq(n^+)^+$. Isto mostra que a tese é verdadeira para $n^+$. Pelo axioma 5, a tese é verdadeira para todo $n$ natural.
		
		\item Tese: $\forall n((n\in\omega)\wedge\nexists m((m\in\omega)\wedge(m^+=n))\to(n=0))$.
		
		Contrapositiva: $\forall n((n\neq0)\to((n\notin\omega)\vee\exists m((m\in\omega)\wedge(m^+=n))))$.
		
		Demonstraremos a contrapositiva, por ser equivalente à tese. Para isto precisaremos da seguinte proposição.
		
		\textbf{Proposição:} Seja $\sigma$ um subconjunto não vazio de $\omega$. Se $0\in\sigma$ e, para cada $n\in\sigma$, valer $n^+\in\sigma$, então $\sigma=\omega$.
		
		\textbf{Prova:} Aplicando o axioma 5 à fórmula $P(\sigma)=(x\in\sigma)$, obtemos $\omega\subset \sigma$. Como $\sigma\subset \omega$, então $\sigma=\omega$.\hfill$\blacksquare$
		
		Seja $\sigma=\{0\}\cup\{n\in\omega:\exists m((m\in\omega)\wedge(m^+=n))\}$. Temos que $0\in\sigma$ e, para cada $n\in\sigma$, $n^+\in\sigma$. Pela proposição acima, $\sigma=\omega$. Pelo axioma 4, $\{0\}\cap\{n\in\omega:\exists m((m\in\omega)\wedge(m^+=n))\}=\emptyset$. Portanto, para qualquer $n\in\omega$ com $n\neq0$, $\exists m((m\in\omega)\wedge(m^+=n))$. Isto demonstra, por contraposição, a tese.
	\end{enumerate}
\end{solucao}

\begin{exercicio}
	Prove que:
	\begin{enumerate}[label=(\alph{*})]
		\item para todo $n\in\omega$, $\emptyset\in n$ ou $\emptyset=n$;
		\item para todos $n,m\in\omega$, se $m\in n$, então $m\subset n$.
	\end{enumerate}
\end{exercicio}
\begin{solucao}
	\begin{enumerate}[label=(\alph{*})]
		\item Para $n=0$, $n=\emptyset$, logo a tese é verdadeira.
		
		Suponha ser verdade para $n$. Se $\emptyset=n$, então $\emptyset\in n^+=n\cup\{n\}$. Se $\emptyset\in n$, então $\emptyset\in n^+$, e a tese é verdadeira para $n^+$
		
		Pelo axioma 5, $\emptyset\in n$ ou $\emptyset=n$ para qualquer $n\in\omega$.
		\item Já feito na prova do Teorema 3.21, item (c).
	\end{enumerate}
\end{solucao}

\begin{exercicio}
	A união de dois conjuntos indutivos é necessariamente um conjunto indutivo? Justifique sua resposta.
\end{exercicio}
\begin{solucao}
	
	\underline{Resposta:} Sim.
	
	\underline{Justificativa:} Sejam $A$ e $B$ dois conjuntos indutivos. Como $\emptyset\in A$ e $\emptyset\in B$, então $\emptyset\in A\cup B$. Se $x\in A\cup B$, então $x\in A$ ou $x\in B$, logo $x^+\in A$ ou $x^+\in B$, o que implica $x^+\in A\cup B$. Isto mostra que $A\cup B$ é indutivo.
\end{solucao}